\documentclass[10pt]{report}
\usepackage{graphicx}
\usepackage[utf8]{inputenc}
\addtolength{\textwidth}{4cm}
\addtolength{\hoffset}{-2cm}
\topmargin -0.30cm
\title{Tarea }
\author{x}
\date{21 de oct}

\begin{document}

\maketitle

\section*{Titulo}
https://viajandopormexico.herokuapp.com/

\subsection*{AngularJS}
AngularJS es un framework de desarrollo web creado por google para codificar l\'{o}gica del lado del cliente. Su arquitectura se basa en MVVC. El framework utiliza javascript como lenguaje principal aunque sus archivos son en su mayor\'{i}a typescript (para integraci\'{o}n con arquitectura JSON).

\subsection*{Algoritmo vor\'{a}z}
El algoritmo utilizado fue una b\'{u}squeda en dos direcciones. Primero se va a considerar las dos ciudades m\'{a}s cercanas al punto de origen, creando una ruta de ida y otra ruta de regreso. De esta manera se calculan las distancias m\'{a}s cortas en un sentido casi circular del recorrido.
\\
\\
1.busquedaVoraz(ubicacionInicial, ubicaciones)

$\;$2.solucion = \emptyset

$\;$3.ordenCiudades = \emptyset

$\;$4.regresoCiudades = \emptyset

$\;$5.ordenCiudades.añadir(menorDistancia(ubicacionInicial,ubicaciones))

$\;$6.ubicaciones = quitarUbicacion(ordenCiudades[0], ubicaciones)

$\;$7.si(ubicaciones != \emptyset)

$\;$
$\;$8.regresoCiudades.añadir(menorDistancia(ubicacionInicial,ubicaciones))

$\;$
$\;$9.ubicaciones = quitarUbicacion(regresoCiudades[0], ubicaciones)

$\;$10.mientras(indice $<$ ubicaciones.longitud)

$\;$
$\;$11.ordenCiudades.añadir(menorDistancia(ubicacionInicial,ubicaciones))

$\;$
$\;$12.ubicaciones = quitarUbicacion(ordenCiudades[indice], ubicaciones)

$\;$
$\;$13.regresoCiudades.añadir(menorDistancia(ubicacionInicial,ubicaciones))

$\;$
$\;$14.ubicaciones = quitarUbicacion(regresoCiudades[indice], ubicaciones)

$\;$15.mientras(indice $<$ regresoCiudades.longitud)

$\;$
$\;$//Añadir primero los \'{u}ltimos t\'{e}rminos

$\;$
$\;$16.ordenCiudades.añadir(regresoCiudades[indice])

$\;$17.solucion = convertirASolucion(ordenCiudades)
\\
\\
Este algoritmo recibe la ubicaci\'{o}n inicial y la lista de ubicaciones seleccionadas por el usuario. Inicialmente se busca la ciudad más cercana al punto inicial, si es la \'{u}nica ciudad seleccionada entonces termina la ruta, en caso contrario va a seleccionar otra ciudad cercana al punto inicial (a esta ruta se le denomina regreso). Este proceso se va a repetir por cada ubicaci\'{o}n que el usuario haya seleccionado hasta que ya no hayan m\'{a}s ubicaciones para elegir. Como se puede observar tiene dos ciclos, el primer ciclo va a recorrer las ubicaciones dadas por el usuario y empezar\'{a} a poblar los dos arreglos (orden de ida y regreso). Este proceso le tomar\'{a} m\'{a}ximo n/2 iteraciones, donde n es el n\'{u}mero de ubicaciones seleccionado, puesto que por cada recorrido se est\'{a}n seleccionando dos objetos del arreglo de ubicaciones. Al finalizar el primer ciclo debe unir los arreglos para dar la soluci\'{o}n final, solamente que el \'{u}ltimo elemento del arreglo de regreso es la continuaci\'{o}n del \'{u}ltimo elemento del arreglo de ida. por lo que la concatenaci\'{o}n debe hacerse al rev\'{e}s. Este proceso se realiza como mucho n/2 veces puesto que la mitad de las ubicaciones ya estan en el arreglo final. Finalmente se añaden la posici\'{o}n inicial al principio y al final del arreglo resultante para aparezcan en la impresi\'{o}n que se le hace al usuario.

Al tener 2 ciclos de n/2, el algoritmo se ejecuta como mucho n veces por lo que su complejidad es O(n).

\end{document}